\chapter{Предметная область}

\section{Описание проблемы}
В среде учащихся и специалистов в области информационных технологий
регулярно проводятся различные соревнования. Формат соревнований предполагает решение задач,
связанных с программированием. Решением задачи является программа, которая проверяется жюри.
Такие соревнования невозможно было бы проводить без автоматизации проверки решений участников,
если участников достаточно много. Более того, автоматизация проверки позволяет проводить соревнования
в большом количестве форматов.

\subsection{Ручной подход}
Жюри после проведения очередного тура соревнования
вручную запускает и оценивает каждое решение участников. Если предположить, что
на проверку одного решения уходит около 5 минут, а тур состоит из 6 задач, то на каждого
участника уйдёт 30 минут человеческого времени. При жюри, состоящем из 3 человек
проверка тура из 40 участников займёт 400 минут (6 часов 40 минут), что недопустимо много.

Такой метод не годится для проведения масштабных соревнований.

\subsection{Полуавтоматический подход}
В этом случае используется частичная автоматизация. Участники по-прежнему сдают решения
после тура, но жюри использует специальную программу для проверки решений.

Чаще всего от решения требуется преобразовать \textit{исходный файл} и получить \textit{результирующий файл}.
Исходный файл принято называть \textbf{тестом}. Путём сравнения результата работы программы
и ответа, имеющегося у жюри, можно оценить качество решения.

Существуют задачи, в которых в тесте допустимыми являются несколько ответов.
В этом случае для сравнения результирующего файла и ответа жюри используется
специальная программа, которую принято называть \textbf{чекером}.

Для проверки решения программа жюри запускает его на каждом тесте
из набора, после чего, если решение успешно завершилось, запускает чекер.
Качество решения оценивается исходя из успешности завершения, потребления ресурсов при работе,
вердикта чекера.

Преимуществом такого подхода по сравнению с предыдущим является существенное повышение скорости
проверки решений. Жюри достаточно запустить проверку всех решений,
присутствие человека не обязательно. Потому на любом очном соревновании не возникнет
проблем со скоростью проверки решений. В случае большого числа участников возможно вести проверку
решений параллельно, в том числе с использованием нескольких компьютеров.

Недостатком такого подхода является невозможность проверить решение на предмет глупых ошибок,
например связанных с форматов ввода-вывода, во время тура. Таким образом,
почти правильное решение может быть не засчитано.

Кроме того, требуется организация передачи решений от участника к жюри после тура.
На этом этапе
\begin{itemize}
    \item решения могут потеряться;
    \item участник может случайно сдать неправильную версию решения;
    \item необходимо обеспечивать идентичность настройки операционных систем и компиляторов
        у участников и жюри, иначе могут возникнуть проблемы с компиляцией решений.
\end{itemize}

\subsection{Автоматический подход}
Жюри предоставляет систему, на базе которой проводится соревнование.
Доступ к системе осуществляется при помощи клиента, в качестве которого
обычно выступает WEB-браузер, но также может использоваться специальный программный продукт.

Участнику предоставляется возможность сдавать решения во время тура. Система жюри
компилирует решение и проверяет на заданном наборе тестов.
В зависимости от правил соревнования система может откладывать проверку на полном наборе тестов,
проверяя решение только на примерах из условия, что защищает участника от глупых ошибок.
Формат отображения результатов также зависит от соревнования. К примеру, результаты
могут отображаться сразу для всех участников, только для авторов решения, отображение результатов
может быть отложено.

При таком подходе жюри вмешивается только если у участников есть претензии к работе системы.

Данный подход позволяет
\begin{itemize}
    \item существенно увеличить гибкость при выборе формата соревнования;
    \item проводить удалённые соревнования -- онлайн-туры;
    \item увеличить количество возможных участников очного тура;
    \item проводить соревнования размещая участников в нескольких различных городах,
        имея при этом общую таблицу результатов.
\end{itemize}

\section{Обзор существующих систем для проведения соревнований}
\subsection{Ejudge}
Ejudge \cite{ejudge} -- это система для проведения различных мероприятий,
в которых необходима автоматическая проверка программ.
Система может применяться для проведения олимпиад и поддержки учебных курсов.

Достоинства:
\begin{itemize}
    \item распространяется в исходных кодах под лицензией GPL;
    \item возможно проведение турниров с автоматической проверкой задач с различными правилами;
    \item легко настраивается.
\end{itemize}

Недостатки:
\begin{itemize}
    \item для полноценной работы требуется модифицированное ядро GNU/Linux,
        при чём специфика модификаций такова, что практически каждое изменение в ядре
        требует работы над патчем;
    \item сложность интеграции новых форматов соревнований, типов задач.
\end{itemize}

\subsection{ACM Server}
Система ACM Server \cite{acmserver} разработана в Ярославском государственном
университете и предназначена для проведения соревнований по программированию на задачах из внешних архивов.
Система представляет собой один исполняемый файл, являющийся web-сервером, и набор плагинов.
Каждый плагин позволяет добавить в турнир задачи с определённого интернет-архива.
При отправке решения система перенаправляет его на внешний архив и следит за состоянием посылки.

Достоинства системы ACM Server:
\begin{itemize}
    \item возможность добавлять в соревнования задачи с популярных архивов задач
        избавляет от необходимости поиска и добавления задач в собственную систему;
    \item развитая система плагинов, позволяющая легко дополнять систему
        собственными модулями работы с внешними архивами задач;
    \item распространяется в исходных кодах под лицензией GPL.
\end{itemize}

Недостатки системы ACM Server:
\begin{itemize}
    \item отсутствие локального архива задач:
        все используемые в соревнованиях задачи должны присутствовать в интернет-архивах;
    \item низкая функциональность, присутствует лишь необходимый минимум для проведения соревнования.
        В целом система не подходит для проведения крупных турниров;
    \item так как система использует одну учетную запись для отправки решений всех участников,
        при активном использовании могут возникнуть проблемы с администрацией определённых архивов.
\end{itemize}

\subsection{Yandex.Contest}
Развивается компанией Yandex.

\subsection{BACS 2.0}
Система BACS \cite{bacs2} развивалась студентами ИжГТУ.
Поддерживает стандартный ACM формат соревнований и школьный формат.
Для проведения школьных соревнований требуется обновлять задачу в архиве до начала и после окончания соревнования,
что недопустимо при проведении регулярных соревнований.
Код не является расширяемым и предназначен только для одной цели -- проведение соревнований в формате ACM.
Backend системы содержит множество уязвимостей, устранение которых требует изменения архитектуры.
Время, требуемое для доработки до приемлимого уровня сравнимо с написанием новой системы, что и было сделано.
Настоящий проект -- продолжение развития проекта BACS, масштабируемый подход к созданию ПО.
