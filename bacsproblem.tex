\section{Задача}
\label{bacsproblem}
Задача -- структурированный набор файлов,
которые позволяют в автоматическом режиме
\begin{itemize}
    \item сгенерировать предоставленные версии условия
        на различных языках и в различных форматах;
    \item получить метаинформацию по задаче, включающую в себя
        \begin{itemize}
            \item имя;
            \item список авторов;
            \item список ответственных;
            \item системная информация, включающая идентификатор пакета тестера;
        \end{itemize}
\end{itemize}

Система поддерживает интеграцию произвольных форматов задач,
на данный момент реализован упрощённый формат задач simple0~\ref{simple0}.

\subsection{Условие}
Каждая задача должна иметь условие. Условие может
быть представлено в нескольких версиях,
различающихся языком и форматом.

\subsection{Тесты}
Для проверки решения используется множество тестов.
Тест имеет уникальный идентификатор внутри задачи,
состоящему из символов
\begin{itemize}
    \item цифры;
    \item маленькие и большие латинские буквы;
    \item символ подчёркивания.
\end{itemize}

Каждый тест представляется набором файлов.
Каждый файл теста определяется идентификатором данных.
Требования к идентификатору аналогичны.
Все тесты из набора должны состоять из одинаковых наборов файлов.

\subsection{Чекер}
При проверке решения возникает необходимость сравнения результатов решения
с ожидаемыми автором. Для этого требуется специальная программа, называемая чекером.

К примеру на соревнованиях ACM~ICPC~\cite{acmicpc} используются задачи,
в которых тесты состоят из входного файла и файла-подсказки для чекера,
который как правило является примером ожидаемого ответа.
Чекеру для такой задачи передаются входной файл, сгенерированный решением выходной,
и подсказка. Анализируя эти три файла чекер выносит вердикт:
\begin{itemize}
    \item \textbf{OK} -- ответ верный;
    \item \textbf{WRONG ANSWER} -- ответ неверный;
    \item \textbf{PRESENTATION ERROR} -- формат ответа неверен,
        к примеру вместо числа обнаружена буква;
    \item \textbf{FAIL TEST} -- обнаружена ошибка теста,
        к примеру решение участника могло получить ответ лучше,
        чем авторское решение. В этом случае требуется
        вмешательство жюри.
\end{itemize}

Чекер как правило представляется в виде отдельной программы.
В этом случае результат возвращается при помощи кода выхода.
Помимо кода выхода чекер может предоставить строковый комментарий для жюри.
Он выводится в стандартный поток вывода или ошибок.

В общем случае чекер -- фукнция,
принимающая на вход тест и множество файлов, сгенерированных решением.
Возвращает пару вердикт и комментарий.

\subsection{Интерактор}
В некоторых задачах существует необходимость проверки взаимодействия решения с окружением.
Такое окружение предоставляет специальная утилита, называемая интерактором.

Задачи, в которых требуется интерактор, принято называть интерактивными~\cite{interactiveproblem}.
На соревнованиях ACM~ICPC~\cite{acmicpc} в интерактивных задачах
решение взаимодействует с интерактором посредством перенаправления
стандартных потоков ввода и вывода с замыканием их на интерактор, см.~рис.~\ref{fig:interactiverun}.

\begin{figure}
    \centering
    \input rs/interactiverun.dot
    \caption{Запуск интерактивной задачи}
    \label{fig:interactiverun}
\end{figure}

\subsection{Формат simple0}
\label{simple0}

\subsubsection{Структура директории}
\begin{itemize}
    \item файл \textbf{format} содержит строку \textbf{bacs/problem/single\#simple0},
        указывающую на формат задачи;
    \item файл \textbf{config.ini} содержит настройки проверки и базовую информацию;
    \item директория \textbf{checker} отвечает за чекер;
    \item директория \textbf{interactor} отвечает за интерактор;
    \item директория \textbf{tests} отвечает за тесты;
    \item директория \textbf{misc} используется для хранения данных,
        используемых при разработке задачи,
        но не используемых системой.
\end{itemize}

\subsubsection{config.ini}
Содержит секции
\begin{itemize}
    \item \textbf{info}
        \begin{itemize}
            \item \textbf{name} -- имя задачи;
            \item \textbf{authors} -- список авторов, разделённый пробелами;
            \item \textbf{maintainers} -- список ответственных, разделённый пробелами;
        \end{itemize}
    \item \textbf{resource\_limits}, где указываются ограничения ресурсов при проверке:
        \begin{itemize}
            \item \textbf{time} время CPU;
            \item \textbf{memory} оперативная память;
            \item \textbf{output} количество выводимых данных;
            \item \textbf{real\_time} реальное время;
        \end{itemize}
    \item \textbf{files}, содержащую настройки перенаправления ввода-вывода:
        \begin{itemize}
            \item \textbf{stdin} указывает имя файла,
                который используется вместо стандартного потока ввода;
            \item \textbf{stdout} указывает имя файла,
                который используется вместо стандартного потока вывода;
            \item \textbf{stderr} указывает имя файла,
                который используется вместо стандартного потока ошибок;
        \end{itemize}
    \item \textbf{tests}, содержащую дополнительную информацию о тестах:
        \begin{itemize}
            \item переменные, начинающиеся с префикса \textbf{group\_}
                содержат списки тестов, составляющих указанную группу.
                К примеру \textbf{group\_pre} описывает группу \textbf{pre}.
        \end{itemize}
\end{itemize}

Пример:
\begin{verbatim}
[info]
name = A + B
maintainers = harmonius

[resource_limits]
time = 1s
memory = 1024MiB
\end{verbatim}

\subsubsection{Чекер и интерактор}
Директория содержит
\begin{itemize}
    \item файл \textbf{config.ini};
    \item файлы исходного кода утилиты.
\end{itemize}

Файл \textbf{config.ini} состоит из секций
\begin{itemize}
    \item \textbf{build}
        \begin{itemize}
            \item \textbf{builder} -- идентификатор сборщика утилиты;
        \end{itemize}
    \item \textbf{utility}
        \begin{itemize}
            \item \textbf{call} -- обёртка для вызова утилиты;
        \end{itemize}
\end{itemize}

Поддерживаемые сборщики и их настройки:
\begin{itemize}
    \item \textbf{single}
        \begin{itemize}
            \item \textbf{source} -- главный исходный файл;
            \item \textbf{libs} -- список требуемых библиотек.
        \end{itemize}
\end{itemize}

Пример файла \textbf{config.ini}:
\begin{verbatim}
[build]
builder = single
source = check.cpp
libs = Lopatin

[utility]
call = in_out_hint
\end{verbatim}

\subsubsection{Условие}
Каждая версия условия указывается в своём \textbf{.ini}-файле
в корне директории условия.

Конфигурационный файл версии условия:
\begin{itemize}
    \item \textbf{info}
        \begin{itemize}
            \item \textbf{lang} -- язык версии;
        \end{itemize}
    \item \textbf{build}
        \begin{itemize}
            \item \textbf{builder} -- сборщик версии условия.
        \end{itemize}
\end{itemize}

Для некоторых типов версии условия требуется процедура сборки.
В этом случае компилятору доступны все файлы всех версий условия,
так как некоторые файлы могут быть общими, например картинки.

Поддерживаемые сборщики и их настройки:
\begin{itemize}
    \item \textbf{copy} -- прямое копирование файлов,
        извлекает формат из суффикса главного файла.
        Опции:
        \begin{itemize}
            \item \textbf{source} -- основной исходный файл;
        \end{itemize}
    \item \textbf{pdflatex} -- создание pdf-файла при помощи утилиты pdflatex.
        Опции:
        \begin{itemize}
            \item \textbf{source} -- основной исходный файл.
        \end{itemize}
\end{itemize}

Пример конфигурационного файла:
\begin{verbatim}
[info]
lang = en

[build]
builder = copy
source = problem.html
\end{verbatim}

\subsubsection{Тесты}
Данная директория состоит из множества файлов.
Каждый файл имеет вид <идентификатор тесты>, точка, <идентификатор данных>.
К примеру \textbf{test.data} -- файл \textbf{data} теста \textbf{test}.
