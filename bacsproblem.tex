\section{Задача}
\label{bacsproblem}
Задача -- структурированный набор файлов,
которые позволяют в автоматическом режиме
\begin{itemize}
    \item сгенерировать предоставленные версии условия
        на различных языках и в различных форматах;
    \item получить метаинформацию по задаче, включающую в себя
        \begin{itemize}
            \item имя;
            \item список авторов;
            \item список ответственных;
            \item системная информация, включающая идентификатор пакета тестера;
        \end{itemize}
\end{itemize}

Система поддерживает различные представления задач.

\subsection{Условие}

\subsection{Тесты}

\subsection{Чекер}

\subsection{Интерактор}

\subsection{Формат simple0}

\subsubsection{Структура директории}
\begin{itemize}
    \item файл \textbf{format} содержит строку \textbf{bacs/problem/single\#simple0},
        указывающую на формат задачи;
    \item файл \textbf{config.ini} содержит настройки проверки и базовую информацию;
    \item директория \textbf{checker} отвечает за чекер;
    \item директория \textbf{interactor} отвечает за интерактор;
    \item директория \textbf{tests} отвечает за тесты;
    \item директория \textbf{misc} используется для хранения данных,
        используемых при разработке задачи,
        но не используемых системой.
\end{itemize}

\subsubsection{config.ini}
Содержит секции
\begin{itemize}
    \item \textbf{info}
        \begin{itemize}
            \item \textbf{name} -- имя задачи;
            \item \textbf{authors} -- список авторов, разделённый пробелами;
            \item \textbf{maintainers} -- список ответственных, разделённый пробелами;
        \end{itemize}
    \item \textbf{resource\_limits}, где указываются ограничения ресурсов при проверке:
        \begin{itemize}
            \item \textbf{time} время CPU;
            \item \textbf{memory} оперативная память;
            \item \textbf{output} количество выводимых данных;
            \item \textbf{real\_time} реальное время;
        \end{itemize}
    \item \textbf{files}, содержащую настройки перенаправления ввода-вывода:
        \begin{itemize}
            \item \textbf{stdin} указывает имя файла,
                который используется вместо стандартного потока ввода;
            \item \textbf{stdout} указывает имя файла,
                который используется вместо стандартного потока вывода;
            \item \textbf{stderr} указывает имя файла,
                который используется вместо стандартного потока ошибок;
        \end{itemize}
    \item \textbf{tests}, содержащую дополнительную информацию о тестах:
        \begin{itemize}
            \item переменные, начинающиеся с префикса \textbf{group\_}
                содержат списки тестов, составляющих указанную группу.
                К примеру \textbf{group\_pre} описывает группу \textbf{pre}.
        \end{itemize}
\end{itemize}

Пример:
\begin{verbatim}
[info]
name = A + B
maintainers = harmonius

[resource_limits]
time = 1s
memory = 1024MiB
\end{verbatim}

\subsubsection{Чекер и интерактор}
Директория содержит
\begin{itemize}
    \item файл \textbf{config.ini};
    \item файлы исходного кода утилиты.
\end{itemize}

Файл \textbf{config.ini} состоит из секций
\begin{itemize}
    \item \textbf{build}
        \begin{itemize}
            \item \textbf{builder} -- идентификатор сборщика утилиты;
        \end{itemize}
    \item \textbf{utility}
        \begin{itemize}
            \item \textbf{call} -- обёртка для вызова утилиты;
        \end{itemize}
\end{itemize}

Поддерживаемые сборщики и их настройки:
\begin{itemize}
    \item \textbf{single}
        \begin{itemize}
            \item \textbf{source} -- главный исходный файл;
            \item \textbf{libs} -- список требуемых библиотек.
        \end{itemize}
\end{itemize}

Пример файла \textbf{config.ini}:
\begin{verbatim}
[build]
builder = single
source = check.cpp
libs = Lopatin

[utility]
call = in_out_hint
\end{verbatim}

\subsubsection{Условие}
Условие представляется в виде нескольких версий,
которые могут отличаться языком и форматом.
Каждая версия условия указывается в своём \textbf{.ini}-файле
в корне директории условия.

Конфигурационный файл версии условия:
\begin{itemize}
    \item \textbf{info}
        \begin{itemize}
            \item \textbf{lang} -- язык версии;
        \end{itemize}
    \item \textbf{build}
        \begin{itemize}
            \item \textbf{builder} -- сборщик версии условия.
        \end{itemize}
\end{itemize}

Поддерживаемые сборщики и их настройки:
\begin{itemize}
    \item \textbf{copy} -- прямое копирование файлов,
        извлекает формат из суффикса главного файла.
        Опции:
        \begin{itemize}
            \item \textbf{source} -- основной исходный файл;
        \end{itemize}
    \item \textbf{pdflatex} -- создание pdf-файла при помощи утилиты pdflatex.
        Опции:
        \begin{itemize}
            \item \textbf{source} -- основной исходный файл.
        \end{itemize}
\end{itemize}

Пример конфигурационного файла:
\begin{verbatim}
[info]
lang = en

[build]
builder = copy
source = problem.html
\end{verbatim}
