\chapter*{Реферат}
Пояснительная записка состоит из
102 страниц, включающих 12 иллюстраций и 1 приложение,
позволяющих наиболее полно составить представление о предметной области и решаемых задачах.
Использовано 11 источников информации.

Ключевые слова: ACM ICPC, контест, backend, распределенная система, проверка решений.

При обучении студентов программированию возникает задача
проверки их знаний. Для этого требуется автоматизированная система
проведения соревнований. Так как количество участников
может быть большим, система должна быть масштабируемой.
Помимо этого необходимо иметь возможность добавлять поддержку
новых типов задач и соревнований, для этого
система должна быть расширяемой.

% TODO методы?

В данной работе подробно рассмотрен вопрос разработки тестирующей системы,
позволяющей разрабатывать пользовательские интерфейсы
для проведения соревнований по программированию.

Разработанная система полностью удовлетворяет поставленной задаче --
является универсальной, расширяемой и масштабируемой.
Успешно применяется для проведения тренировок студентов
и проведения лабораторных работ.

Модульность системы позволяет независимо изменять различные её компоненты,
связь между которыми осуществляется посредством технологии RPC.
Благодаря этому можно полностью заменить реализацию компонентов,
что не повлияет на работы системы.

Горизонтальная масштабируемость реализована прозрачно,
что даёт возможность добавлять дополнительные
вычислительные мощности не изменяя конфигурацию вычислительной сети.
В любой момент времени существует возможность включить машину в кластер
и исключить из него. Это не влияет на работу системы.

Система использует специально разработанную песочницу для
безопасного запуска внешнего непроверенного кода и контроля потребления ресурсов.
Это позволяет снизить риск взлома системы извне.
Песочница успешно применяется в компании ООО «Яндекс» в проекте Яндекс.Контест.
