\documentclass[xetex,mathserif,serif,10pt]{beamer}
%\documentclass[11pt]{article}
\usepackage{xltxtra}
\usepackage{polyglossia}
\setdefaultlanguage[spelling=modern]{russian}
%\setmainfont[Mapping=tex-text]{DejaVu Sans}
%\setmainfont[Mapping=tex-text]{Liberation Sans}
\setmonofont[Mapping=tex-text]{DejaVu Sans Mono}
\setmainfont[Mapping=tex-text]{Linux Libertine O}
%\setmonofont[Mapping=tex-text]{Liberation Mono}

\input dot2tex

\usepackage{verbatim}
\usepackage{tabularx}
\usepackage{float}
\usepackage{url}

\usepackage{textpos}

\usepackage{hyperref}

\usepackage{indentfirst}

\usepackage{algorithm}
\usepackage{algorithmic}

%\setbeamertemplate{caption}[numbered]
\setbeamertemplate{footline}[frame number]

\input config

%\usetheme{Warsaw}
\usetheme{Singapore}

\newenvironment{sframe}[2]{\section{#1}\begin{frame}[label=#2]{#1}}{\end{frame}}

\addtobeamertemplate{headline}{}{
\begin{textblock*}{100mm}(.015\textwidth,0.2cm)
\includegraphics[height=1cm,width=1cm]{logo}
\end{textblock*}}

\begin{document}
    \title[BACS]{BACS ACM Contest System}
    \author[Филиппов]{А.~Филиппов\inst{1}}
    \institute
    {
        \inst{1}
        Ижевский государственный технический университет имени М.Т.~Калашникова
    }
    %\frame{\titlepage}

    \input beamertitle.tex

    \begin{sframe}{Цель работы}{target}
        Разработать тестирующую систему как основу для разработки
        автоматических систем проведения соревнований по программированию.

        \begin{itemize}
            \item API;
            \item Расширяемость;
            \item Модульность.
        \end{itemize}
    \end{sframe}

    \begin{sframe}{Задачи работы}{problems}
        \begin{itemize}
            \item \hyperlink{systemdesign}{Проектирование системы};
            \item \hyperlink{dcs}{Разработка распределённой системы};
            \item \hyperlink{bunsanpm}{Разработка пакетного менеджера};
            \item \hyperlink{bacsarchive}{Разработка архива задач};
            \item \hyperlink{bacsproblem}{Разработка форматов задач};
            \item \hyperlink{bacsstatementprovider}{Разработка сервиса условий}.
        \end{itemize}
    \end{sframe}

    \begin{sframe}{Архитектура системы}{systemdesign}
        \begin{figure}
            \resizebox{\columnwidth}{!}{
                \input systemdesign.dot
            }
        \end{figure}
    \end{sframe}

    \begin{sframe}{Распределённая система}{dcs}
        \begin{itemize}
            \item Сервис регистрируется;
            \item Клиент получает URI сервиса по известному идентификатору;
            \item Клиент обращается к сервису.
        \end{itemize}
        \begin{figure}
            \centering
            \includegraphics[width=\columnwidth]{bunsandcs}
        \end{figure}
    \end{sframe}

    \begin{sframe}{Пакетный менеджер}{bunsanpm}
        \begin{itemize}
            \item Удалённый репозиторий;
            \item Кэширование промежуточных результатов;
            \item Компиляция на целевой машине.
        \end{itemize}
        \begin{figure}
            \centering
            \includegraphics[width=\columnwidth]{bunsanpm}
        \end{figure}
    \end{sframe}

    \begin{sframe}{Архив задач}{bacsarchive}
        \begin{itemize}
            \item Пользователь добавляет задачу в архив;
            \item Архив преобразует задачу из пользовательского формата во внутренний,
                помещая данные в репозиторий;
            \item Пользователю предоставляется доступ к информации о задаче,
                извлечённой в процессе преобразования.
        \end{itemize}
        \begin{figure}
            \centering
            \includegraphics[width=\columnwidth]{bacsarchive}
        \end{figure}
    \end{sframe}

    \begin{sframe}{Формат задачи}{bacsproblem}
        \begin{itemize}
            \item Поддержка пользовательских форматов задач;
            \item Лёгкость добавления поддержки новых форматов;
            \item Преобразование в универсальный внутренний формат.
        \end{itemize}
    \end{sframe}

    \begin{sframe}{Сервис условий}{bacsstatementprovider}
        \begin{itemize}
            \item Пользователь запрашивает условие у WEB-интерфейса;
            \item WEB-интерфейс переадресует запрос сервису условий;
            \item Сервис условий генерирует условие из исходного кода,
                находящегося в репозитории;
            \item Сервис условий предоставляет пользователю доступ к условию.
        \end{itemize}
        \begin{figure}
            \centering
            \includegraphics[width=\columnwidth]{bacsstatementprovider}
        \end{figure}
    \end{sframe}

    \begin{frame}
        \Large\centering Спасибо за внимание!
    \end{frame}

    \begin{frame}{Содержание}
        \tableofcontents
    \end{frame}
\end{document}
