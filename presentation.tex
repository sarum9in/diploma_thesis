\documentclass[xetex,mathserif,serif,10pt]{beamer}
%\documentclass[11pt]{article}
\usepackage{xltxtra}
\usepackage{polyglossia}
\setdefaultlanguage[spelling=modern]{russian}
%\setmainfont[Mapping=tex-text]{DejaVu Sans}
%\setmainfont[Mapping=tex-text]{Liberation Sans}
\setmonofont[Mapping=tex-text]{DejaVu Sans Mono}
\setmainfont[Mapping=tex-text]{Linux Libertine O}
%\setmonofont[Mapping=tex-text]{Liberation Mono}

\input dot2tex

\usepackage{verbatim}
\usepackage{tabularx}
\usepackage{float}
\usepackage{url}

\usepackage{textpos}

\usepackage{hyperref}

\usepackage{indentfirst}

\usepackage{algorithm}
\usepackage{algorithmic}

\usepackage{graphicx}

%\setbeamertemplate{caption}[numbered]
\setbeamertemplate{footline}[frame number]

\input config

%\usetheme{Warsaw}
\usetheme{Singapore}

\newenvironment{sframe}[2]{\section{#1}\begin{frame}[label=#2]{#1}}{\end{frame}}

\addtobeamertemplate{headline}{}{
\begin{textblock*}{100mm}(.015\textwidth,0.2cm)
\includegraphics[height=1cm,width=1cm]{rs/logo}
\end{textblock*}}

\begin{document}
    \title[BACS]{BACS ACM Contest System}
    \author[Филиппов]{А.~Филиппов\inst{1}}
    \institute
    {
        \inst{1}
        Ижевский государственный технический университет имени М.Т.~Калашникова
    }
    %\frame{\titlepage}

    \input beamertitle.tex

    \begin{sframe}{Цель работы}{target}
        \begin{block}{Цель}
            Разработать тестирующую систему как основу для разработки
            автоматизированных систем проведения соревнований по программированию
        \end{block}

        \begin{block}{Требования}
            \begin{itemize}
                \item API -- программный интерфейс
                \item Расширяемость
                \item Модульность
            \end{itemize}
        \end{block}
    \end{sframe}

    \begin{sframe}{Задачи работы}{problems}
        \begin{itemize}
            \item \hyperlink{systemdesign}{Проектирование тестирующей системы}
            \item \hyperlink{dcs}{Разработка системы распределения нагрузки}
            \item \hyperlink{bunsanpm}{Разработка пакетного менеджера}
            \item \hyperlink{bacsarchive}{Разработка архива задач}
            \item \hyperlink{bacsproblem}{Разработка форматов задач}
            \item \hyperlink{bacsstatementprovider}{Разработка сервиса условий}
            \item \hyperlink{sandbox}{Разработка песочницы}
        \end{itemize}
    \end{sframe}

    \begin{sframe}{Архитектура системы}{systemdesign}
        \begin{block}{}
            \begin{figure}[H]
                \centering
                \includegraphics{rs/systemdesign}
            \end{figure}
        \end{block}
    \end{sframe}

    \begin{sframe}{Система распределения нагрузки}{dcs}
        \begin{itemize}
            \item Сервис регистрируется
            \item Клиент получает URI сервиса по известному идентификатору
            \item Клиент обращается к сервису
        \end{itemize}
        \begin{figure}
            \centering
            \includegraphics[width=\columnwidth]{rs/bunsandcs}
        \end{figure}
    \end{sframe}

    \begin{frame}{Система распределения нагрузки: интеграция}
        \begin{figure}
            \centering
            \includegraphics[width=\columnwidth]{rs/bunsandcsbacs}
        \end{figure}
    \end{frame}

    \begin{sframe}{Пакетный менеджер}{bunsanpm}
        \begin{itemize}
            \item Удалённый репозиторий
            \item Кэширование промежуточных результатов
            \item Компиляция на целевой машине
        \end{itemize}
        \begin{figure}
            \centering
            \includegraphics[width=\columnwidth]{rs/bunsanpm}
        \end{figure}
    \end{sframe}

    \begin{sframe}{Архив задач}{bacsarchive}
        \begin{itemize}
            \item Пользователь добавляет задачу в архив в одном из поддерживаемых форматов.
            \item Архив преобразует задачу из пользовательского формата во внутренний --
                множество пакетов, которые помещаются в репозиторий.
            \item Пользователю предоставляется доступ к информации о задаче
                (название, авторы, отвественные, рекомендуемые настройки проверки, ...).
        \end{itemize}
        \begin{figure}
            \centering
            \includegraphics[width=\columnwidth]{rs/bacsarchive}
        \end{figure}
    \end{sframe}

    \begin{sframe}{Форматы задач}{bacsproblem}
        \begin{itemize}
            \item Поддержка пользовательских форматов задач
            \item Лёгкость добавления поддержки новых форматов
            \item Преобразование в универсальный внутренний формат
        \end{itemize}
    \end{sframe}

    \begin{sframe}{Сервис условий}{bacsstatementprovider}
        \begin{itemize}
            \item Пользователь запрашивает условие у WEB-интерфейса
            \item WEB-интерфейс переадресует запрос сервису условий
            \item Сервис условий генерирует условие из исходного кода,
                находящегося в репозитории
            \item Сервис условий предоставляет пользователю доступ к условию
        \end{itemize}
        \begin{figure}
            \centering
            \includegraphics[width=\columnwidth]{rs/bacsstatementprovider}
        \end{figure}
    \end{sframe}

    \begin{sframe}{Тестер, Песочница}{sandbox}
        \begin{itemize}
            \item Предоставление чистой среды
            \item Изоляция
            \item Контроль ресурсов
        \end{itemize}
        \begin{figure}
            \resizebox{\columnwidth}{!}{
                \input rs/yandexcontestinvoker.dot
            }
        \end{figure}
    \end{sframe}

    \begin{frame}
        \Large\centering Спасибо за внимание!
    \end{frame}

    \begin{frame}{Содержание}
        \tableofcontents
    \end{frame}
\end{document}
