\section{Архив задач}
\subsection{Назначение}
Архив задач -- сервис для хранения и обработки задач.

Для того, чтобы обеспечить тестирование пользовательских решений,
системе необходимо иметь доступ к материалам для тестирования.
Их предоставляют авторы задачи. В их число входят
\begin{itemize}
    \item тесты;
    \item специальные вспомогательные программы, такие как
        \begin{itemize}
            \item чекер;
            \item интерактор;
            \item валидатор тестов;
        \end{itemize}
    \item настройки запуска решения;
    \item условие задачи, возможно в нескольких версиях: различные форматы и языки.
\end{itemize}

Как можно заметить, часть данных задачи используется WEB-интерфейсом,
другая самой тестирующей системой.

Одной из главных задач архива является хранение задачи
в неизменном виде. Вне зависимости от того, что добавляет
в архив пользователь, это будет сохранено. В случае несоответствие
данных установленному формату будет предоставлена диагностическая информация
с указанием возможной проблемы. Такая задача не используется системой и требует
доработки. Пользователь имеет возможность скачать исходные материалы задачи
для работы над ней.

\subsection{API}
Сервис предоставляет программный интерфейс посредством
HTTP~POST RPC-интерфейса. Ответы возвращаются в формате Protocol Buffers.

\subsubsection{Функции API}
\begin{itemize}
    \item \textbf{insert}, операция записи, добавляет задачи из отправленного архива;
    \item \textbf{extract}, операция чтения, извлекает указанные задачи, отправляя в архиве;
    \item \textbf{rename}, операция записи, переименовывает задачу;
    \item \textbf{existing}, операция чтения, проверяет задачи на существование;
    \item \textbf{available}, операция чтения, проверяет задачи на доступность для тестирования;
    \item \textbf{status}, операция чтения, возвращает флаги и хэш задач;
    \item \textbf{with flag}, операция чтения, проверяет наличие указанных флагов у задач;
    \item \textbf{set flags}, операция записи, устанавливает указанные флаги задачам;
    \item \textbf{unset flags}, операция записи, удаляет указанные флаги у задач;
    \item \textbf{clear flags}, операция записи, удаляет все флаги у задач;
    \item \textbf{ignore}, операция записи, устанавливает флаг недоступности задач для тестирования;
    \item \textbf{info}, операция чтения, возвращает метаинформацию о задаче;
    \item \textbf{hash}, операция чтения, возвращает хэш задач;
    \item \textbf{repack}, операция записи, запускает повторные анализ задачи,
        может потребоваться при обновлении ПО.
\end{itemize}

\subsubsection{Атомарность API}
Так как доступ к API могут иметь несколько приложений враз,
необходимо обеспечение безопасности операций от коллизий.
Эксклюзивный доступ на запись в архив в каждый момент времени
имеет максимум 1 клиент. Доступ на чтение могут иметь
множество клиентов враз (при отсутствие клиента
с доступом на запись).

Так как задача может быть обновлена без уведомления приложения,
для обновлений информации используется хэш задачи.
Гарантируется, что изменение информации о задачи
влечёт за собой изменение её хэша. Таким образом,
получив информацию о задаче,
клиент в дальнейшем может сверять её хэш и запрашивать
информацию заново только в случае его изменения.

\subsection{Внутреннее устройство}
Сервис построен на основе framework'а CppCMS~\cite{cppcms}.
Функционал извлечения информации из задач и конвертирования
в другие форматы реализуется посредством подключаемых библиотек, см.~\ref{bacsproblem}.

\subsection{Формат внутреннего хранилища}
Данные сохраняются в определённой для этого директории.
Для каждой задачи отводится своя поддиректория, содержащая
\begin{itemize}
    \item архив с исходными данными задачи;
    \item директорию \textbf{flags}, содержащую набор файлов,
        имена которых являются установленными флагами;
    \item файл \textbf{hash}, содержащий хэш задачи;
    \item файл \textbf{info}, содержащий метаинформацию о задаче.
\end{itemize}
