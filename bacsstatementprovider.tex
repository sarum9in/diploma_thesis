\section{Сервис условий}
Так как для создания условия может требоваться процедура сборки
был разработан сервис, предоставляющий прозрачный доступ к условиям задач.

Для доступа к условию WEB-интерфейс формирует специальный URL,
содержащий идентификатор задачи и версии условия, и перенаправляет
пользователя на него.

Вид URL'а: \verb=http[s]://{host}/{referrer}/get/{statement}/{path}=,
где
\begin{itemize}
    \item \textbf{host} -- адрес сервиса условий;
    \item \textbf{referrer} -- идентификатор WEB-интерфейса;
    \item \textbf{statement} -- закодированный идентификатор версии условия;
    \item \textbf{path} -- путь до файла внутри условия.
\end{itemize}

В случае, если компонент \textbf{path} отсутствует,
сервис перенаправляет пользователя на index-файл версии условия.
Это актуально для форматов, состоящих из нескольких файлов.
В них всегда присутствует файл, являющийся точкой входа
и ссылающийся на все остальные файлы.
К примеру для формата HTML это основная страница,
а дополнительные файлы могут быть картинками или скриптами.

Идентификатор версии условия кодируется для того,
чтобы пользователь не мог определить внутренний
идентификатор задачи. Помимо этого
в данную структуру помещается хэш задачи
для того, чтобы при обновлении задачи
пользователь не смог автоматически получить доступ к ней.
Права доступа к задаче обновит WEB-интерфейс.
