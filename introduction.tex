\chapter*{Введение}
\addcontentsline{toc}{chapter}{Введение}
В настоящее время приобретают популярность системы автоматизации проведения
олимпиад, конкурсов, лабораторных работ. В структуре таких систем всегда
можно выделить две части -- пользовательский интерфейс и систему тестирования.

Автоматизированные системы тестирования позволяют организовывать удалённое обучение,
снизить нагрузку на преподавателя при очном обучении во время
проведения лабораторных и контрольных работ,
позволяют проводить курсы по подготовке школьников и студентов к олимпиадам.

Система тестирования является важным компонентом,
который отвечает за безопасность всей системы. Именно здесь
происходит работа с программным кодом, предоставленным внешними
пользователями. Такой код не должен иметь возможности
нарушить работу системы, помимо этого должна быть обеспечена
возможность автоматической оценки соответствия его заданным
критериям.

Целью работы является создание системы,
которая может быть использована в качестве фундамента
для надстройки различных пользовательских интерфейсов
систем автоматизации проведения соревнований по программированию.
