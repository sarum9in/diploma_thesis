\section{Безопасный запуск внешнего кода на сервере, Invoker}
При запуске стороннего кода всегда
существует опасность взлома системы.
Помимо этого при проверке решений необходимо
оценивать их эффективность, измеряя потребляемые ресурсы.

Для этого используется библиотека,
разработанная в ходе летней практики в компании ООО~<<Яндекс>>.
Она позволяет гибко настраивать параметры запуска,
изолировать программный код, измерять затрачиваемое процессорное время,
оперативную память, реальное время.

Библиотека спроектирована для работы на операционной системе GNU/Linux.
Для изоляции используется технология пространств~имён~\cite{linuxns} ядра.
Перед запуском процесса библиотека создаёт отдельные пространства
имён для файловой системы, пространства процессов, сети. Таким образом,
процесс оказывается изолирован от основной системы.
Помимо этого процесс запускается с правами непривилегированного пользователя,
что сильно снижает вероятность выхода из изолированного окружения.

Для контроля потребляемых ресурсов используются контрольные~группы~\cite{linuxcgroups}
ядра операционной системы.
